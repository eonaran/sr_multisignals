\documentclass[11pt]{article}

% Packages
\usepackage{amscd}
\usepackage{amsmath}
\usepackage{amstext}
\usepackage{bbold}
\usepackage{bm}
\usepackage{booktabs}
\usepackage{color}
\usepackage{easybmat}
\usepackage{etex}
\usepackage{framed}
\usepackage[dvips,letterpaper,margin=1in]{geometry}
\usepackage{graphicx}
\usepackage{hyperref}
\usepackage[noabbrev,capitalize]{cleveref}
\usepackage{mathtools}
\usepackage{setspace}
\usepackage{verbatim}

%
% Commands
%

% Figures
\newcommand{\fig}[1]{(figure \ref{#1})}
\newcommand{\FIG}[1]{figure \ref{#1}}

\DeclareSymbolFont{bbold}{U}{bbold}{m}{n}
\DeclareSymbolFontAlphabet{\mathbbold}{bbold}

% Names
\newcommand{\Mobius}{M\"{o}bius}
\newcommand{\Holder}{H\"{o}lder}
\newcommand{\Rouche}{Rouch\'{e}}
\newcommand{\Ito}{It\={o}}
\newcommand{\Kondo}{Kond\^{o}}
\newcommand{\Levy}{L\'{e}vy}
\newcommand{\Cramer}{Cram\'{e}r}
\newcommand{\Godel}{G\"{o}del}
\newcommand{\Carath}{Carath\'{e}odory}
\newcommand{\Caratheodory}{Carath\'{e}odory}
\newcommand{\Hopital}{H\^{o}pital}

% Random
\renewcommand{\bar}{\overline}
\newcommand{\lvec}{\overrightarrow}
\newcommand{\ra}{\rangle}
\newcommand{\la}{\langle}

% Disjoint union
\makeatletter
\def\moverlay{\mathpalette\mov@rlay}
\def\mov@rlay#1#2{\leavevmode\vtop{%
   \baselineskip\z@skip \lineskiplimit-\maxdimen
   \ialign{\hfil$\m@th#1##$\hfil\cr#2\crcr}}}
\newcommand{\charfusion}[3][\mathord]{
    #1{\ifx#1\mathop\vphantom{#2}\fi
        \mathpalette\mov@rlay{#2\cr#3}
      }
    \ifx#1\mathop\expandafter\displaylimits\fi}
\makeatother

\newcommand{\cupdot}{\charfusion[\mathbin]{\cup}{\cdot}}
\newcommand{\bigcupdot}{\charfusion[\mathop]{\bigcup}{\cdot}}

% Blackboard bold
\newcommand{\RR}{\mathbb{R}}
\newcommand{\QQ}{\mathbb{Q}}
\newcommand{\NN}{\mathbb{N}}
\newcommand{\TT}{\mathbb{T}}
\newcommand{\ZZ}{\mathbb{Z}}
\newcommand{\DD}{\mathbb{D}}
\newcommand{\HH}{\mathbb{H}}
\newcommand{\CC}{\mathbb{C}}
\newcommand{\PP}{\mathbb{P}}
\newcommand{\EE}{\mathbb{E}}
\renewcommand{\AA}{\mathbb{A}}
\newcommand{\FF}{\mathbb{F}}
\renewcommand{\SS}{\mathbb{S}}
\newcommand{\Fp}{\FF_p}
\newcommand{\TrivGp}{\mathbbold{1}}
\newcommand{\One}{\mathbbold{1}}

\newcommand{\RP}{\RR\mathrm{P}}
\newcommand{\CP}{\CC\mathrm{P}}

% Vector bold
\newcommand{\nn}{\bm{n}}
\newcommand{\vv}{\bm{v}}
\newcommand{\ww}{\bm{w}}
\newcommand{\xx}{\bm{x}}
\newcommand{\yy}{\bm{y}}
\renewcommand{\vec}[1]{\mathbf{#1}}
\newcommand{\one}{\bm{1}}

% Other fonts
\newcommand{\fp}{\mathfrak{p}}
\newcommand{\fq}{\mathfrak{q}}
\newcommand{\fg}{\mathfrak{g}}
\newcommand{\fh}{\mathfrak{h}}
\newcommand{\fa}{\mathfrak{a}}
\newcommand{\fb}{\mathfrak{b}}
\newcommand{\fc}{\mathfrak{c}}
\newcommand{\fm}{\mathfrak{m}}
\renewcommand{\sl}{\mathfrak{sl}}
\newcommand{\so}{\mathfrak{so}}
\newcommand{\gl}{\mathfrak{gl}}
\renewcommand{\sp}{\mathfrak{sp}}
\newcommand{\sA}{\mathcal{A}}
\newcommand{\sB}{\mathcal{B}}
\newcommand{\sC}{\mathcal{C}}
\newcommand{\sD}{\mathcal{D}}
\newcommand{\sE}{\mathcal{E}}
\newcommand{\sF}{\mathcal{F}}
\newcommand{\sG}{\mathcal{G}}
\newcommand{\sH}{\mathcal{H}}
\newcommand{\sI}{\mathcal{I}}
\newcommand{\sL}{\mathcal{L}}
\newcommand{\sM}{\mathcal{M}}
\newcommand{\sN}{\mathcal{N}}
\newcommand{\sO}{\mathcal{O}}
\newcommand{\sP}{\mathcal{P}}
\newcommand{\sR}{\mathcal{R}}
\newcommand{\sS}{\mathcal{S}}
\newcommand{\sT}{\mathcal{T}}
\newcommand{\sU}{\mathcal{U}}
\newcommand{\sV}{\mathcal{V}}
\newcommand{\sX}{\mathcal{X}}
\newcommand{\sY}{\mathcal{Y}}

% Spacing
\newcommand\ThmBr{%
    \@ifstar{\item[\vbox{\null}]}{%
      \begingroup % keep changes local
      \setlength\itemsep{0pt}%
      \setlength\parsep{0pt}%
       \item[\vbox{\null}]%
      \endgroup%
     }}
\newcommand{\br}{\vspace{1pc}}
\newcommand{\BR}{\vspace{2pc}}
\newcommand{\picspace}{\vspace{13pc}}
\newcommand{\hs}{\hspace{1mm}}
\newcommand{\HS}{\hspace{3.5mm}}
\newcommand{\hr}{
  \begin{center}
    \line(1,0){250}
  \end{center}
}
\newcommand{\hrs}{
  \begin{center}
    \line(1,0){150}
  \end{center}
}

% Plain text
\renewcommand{\Re}{\mathrm{Re}}
\renewcommand{\Im}{\mathrm{Im}}

\newcommand{\Var}{\mathrm{Var}}
\newcommand{\Cov}{\mathrm{Cov}}

\newcommand{\disc}{\mathrm{disc}}
\newcommand{\Ann}{\mathrm{Ann}}
\newcommand{\Ass}{\mathrm{Ass}}
\newcommand{\Soc}{\mathrm{Soc}}
\newcommand{\Supp}{\mathrm{Supp}}
\newcommand{\Spec}{\mathrm{Spec}}
\newcommand{\maxSpec}{\mathrm{maxSpec}}

\newcommand{\N}{\mathrm{N}}
\newcommand{\Tr}{\mathrm{Tr}}

% Functors
\newcommand{\Hom}{\mathrm{Hom}}
\newcommand{\Der}{\mathrm{Der}}
\newcommand{\End}{\mathrm{End}}
\newcommand{\Ind}{\mathrm{Ind}}
\newcommand{\Aut}{\mathrm{Aut}}
\newcommand{\Gal}{\mathrm{Gal}}
\newcommand{\Sym}{\mathrm{Sym}}
\newcommand{\Rad}{\mathrm{Rad}}
\newcommand{\Id}{\mathrm{Id}}
\newcommand{\Ad}{\mathrm{Ad}}
\newcommand{\ad}{\mathrm{ad}}
\newcommand{\Pow}{\mathrm{Pow}}
\newcommand{\diam}{\mathrm{diam}}

\newcommand{\img}{\mathrm{img}}
\newcommand{\sgn}{\mathrm{sgn}}

\newcommand{\ch}{\mathrm{char}}
\newcommand{\Res}{\mathrm{Res}}
\newcommand{\ord}{\mathrm{ord}}
\newcommand{\cont}{\mathrm{cont}}
\newcommand{\ab}{\mathrm{ab}}
\newcommand{\Orb}{\mathrm{Orb}}
\newcommand{\Syl}{\mathrm{Syl}}
\newcommand{\Irr}{\mathrm{Irr}}
\newcommand{\Frac}{\mathrm{Frac}}
\newcommand{\sep}{\mathrm{sep}}
\newcommand{\per}{\mathrm{per}}

% Groups
\newcommand{\GL}{\mathrm{GL}}
\newcommand{\PGL}{\mathrm{PGL}}
\newcommand{\PSL}{\mathrm{PSL}}
\newcommand{\SL}{\mathrm{SL}}
\newcommand{\oO}{\mathrm{O}}
\newcommand{\SO}{\mathrm{SO}}
\newcommand{\PSO}{\mathrm{PSO}}
\newcommand{\Sp}{\mathrm{Sp}}
\newcommand{\PSp}{\mathrm{PSp}}
\newcommand{\U}{\mathrm{U}}
\newcommand{\SU}{\mathrm{SU}}
\newcommand{\PSU}{\mathrm{PSU}}

% Parentheses
\newcommand{\lgndr}[2]{\ensuremath{\left(\frac{#1}{#2}\right)}}

% Mappings
\newcommand{\iso}{\cong}
\newcommand{\eqdf}{\stackrel{\mathrm{df}}{=}}
\newcommand{\eqd}{\stackrel{\mathrm{d}}{=}}
\newcommand{\eqqu}{\stackrel{\mathrm{?}}{=}}
\newcommand{\xto}{\xrightarrow}
\newcommand{\dto}{\Rightarrow}
\newcommand{\into}{\hookrightarrow}
\newcommand{\xinto}{\xhookrightarrow}
\newcommand{\onto}{\twoheadrightarrow}
\newcommand{\xonto}{xtwoheadrightarrow}
\newcommand{\isoto}{\xto{\sim}}
\newcommand{\upto}{\nearrow}
\newcommand{\downto}{\searrow}

% Convenience
\newcommand{\Implies}{\ensuremath{\Rightarrow}}
\newcommand{\ImpliedBy}{\ensuremath{\Leftarrow}}
\newcommand{\Iff}{\ensuremath{\Leftrightarrow}}

\newcommand{\Pfright}{\ensuremath{(\Rightarrow):\hs}}
\newcommand{\Pfleft}{\ensuremath{(\Leftarrow):\hs}}

\newcommand{\sm}{\ensuremath{\setminus}}

\newcommand{\tab}[1]{(table \ref{#1})}
\newcommand{\TAB}[1]{table \ref{#1}}

\newcommand{\precode}[1]{\textbf{\footnotesize #1}}
\newcommand{\code}[1]{\texttt{\footnotesize #1}}

\newcommand{\sectionline}{
  \nointerlineskip \vspace{\baselineskip}
  \hspace{\fill}\rule{0.35\linewidth}{.7pt}\hspace{\fill}
  \par\nointerlineskip \vspace{\baselineskip}
}


%
% Misc
%

\parskip0em
\linespread{1.05}
\widowpenalty10000
\clubpenalty10000


\newcommand{\HC}{\mathsf{HC}}
\newcommand{\CHC}{\mathsf{CHC}}
\newcommand{\SK}{\mathsf{SK}}
\newcommand{\SDP}{\mathsf{SDP}}
\newcommand{\SOS}{\mathsf{SOS}}
\newcommand{\PE}{\mathsf{PE}}
\newcommand{\PS}{\mathsf{PS}}
\newcommand{\tEE}{\tilde{\mathbb{E}}}
\newcommand{\tCov}{\widetilde{\mathrm{Cov}}}
\newcommand{\sfP}{\mathsf{P}}
\newcommand{\argmin}{\mathrm{argmin}}
\newcommand{\argmax}{\mathrm{argmax}}
\newcommand{\diag}{\mathrm{diag}}
\newcommand{\vrad}{\mathrm{vrad}}
\newcommand{\rk}{\mathrm{rk}}
\newcommand{\rkeff}{\mathrm{rk}_{\mathrm{eff}}}
\newcommand{\GOE}{\mathsf{GOE}}
\newcommand{\ssG}{\mathsf{G}}
\newcommand{\bA}{\bm A}
\newcommand{\bB}{\bm B}
\newcommand{\bD}{\bm D}
\newcommand{\bG}{\bm G}
\newcommand{\bH}{\bm H}
\newcommand{\bP}{\bm P}
\newcommand{\bQ}{\bm Q}
\newcommand{\bR}{\bm R}
\newcommand{\bW}{\bm W}
\newcommand{\bX}{\bm X}
\newcommand{\ba}{\bm a}
\newcommand{\bb}{\bm b}
\newcommand{\bg}{\bm g}
\newcommand{\bv}{\bm v}
\newcommand{\bw}{\bm w}
\newcommand{\bx}{\bm x}

\DeclareRobustCommand{\bmrob}[1]{\bm{#1}}
\pdfstringdefDisableCommands{%
  \renewcommand{\bmrob}[1]{#1}%
}

\newtheorem{theorem}{Theorem}
\newtheorem{lemma}{Lemma}
\newtheorem{question}{Question}
\newtheorem{definition}{Definition}
\newtheorem{example}{Example}
\newtheorem{proposition}{Proposition}
\newtheorem{corollary}{Corollary}
\newtheorem{conjecture}{Conjecture}

\title{Super-Resolution of Multiple Signals with Common Support}

\begin{document}

\maketitle

\noindent

\section{Introduction}

\subsection{Single Observation Recovery}

We are interested in the following problem: a signal is observed at a finite set of times $t_j \in T \subset [0, 1]$ with amplitudes $a_j \in \CC$, forming a time-domain ideal observation
\[ x = \sum_{t_j \in T} a_j\delta_{t_j}. \]
If the measurements are low-resolution, the actual observation may be modeled by convolving with a low-pass point spread function (PSF) $\phi$, giving
\[ x_L = (\phi * x)(t) = \sum_{t_j \in T} a_j \phi(t - t_j). \]
It is convenient to simplify and assume that in the Fourier domain $\phi$ is a constant supported on $[-f_c, f_c]$, and that what we really observe is finitely many samples of the Fourier transform of $x$,
\[ y(k) = (\sF_{[-f_c, f_c]} x_L)(k) = \int_0^1 e^{-2\pi i k t} x_L(t) dt = \sum a_j e^{-2\pi i k t_j} \]
for $k \in \ZZ \cap [-f_c, f_c]$.

\subsection{Dual Certificate}

\begin{definition}
    Let $\mu \subset \CC$ denote the complex unit circle.
    A \emph{sign pattern} on a set is an assignment of points of $\mu$ to each point of the set.
\end{definition}

\begin{definition}
    For a sign pattern $v \in \mu^T$, a low-pass trigonometric polynomial $q: \TT \to \CC$ given by
    \[ q(t) = \sum_{k = -f_c}^{f_c} c_k e^{2\pi i k t} \]
    is a \emph{(single-observation) dual certificate} for $v$ if $q(t_j) = v_j$ for $t_j \in T$, and $|q(t)| < 1$ for $t \notin T$.
\end{definition}
If a dual certificate for the sign pattern $v = \frac{a_j}{|a_j|}$ exists, it follows that $x$ is the unique solution of TV minimization for the super-resolution problem.
Therefore, to prove the effectiveness of TV minimization it suffices to show that a dual certificate exists under some conditions on $T$.
The main result of \cite{fernandez2016super} is thus equivalent to the following.
\begin{theorem}[Proposition 2.3 of \cite{fernandez2016super}]
    If $\Delta(T) \geq \frac{1.26}{f_c}$ and $f_c \geq 10^3$, then a dual certificate exists for any sign pattern on $T$.
    \label{thm:single-obs-recovery}
\end{theorem}

\subsection{Multiple Observation Recovery}

\section{Single-Observation Recovery Proof}

In this section, we outline the proof from \cite{fernandez2016super} of Theorem~\ref{thm:single-obs-recovery}.
It is most intuitive to view the dual certificate as the result of a sequence of increasingly more and more refined constructions, noting at each point what does not suffice about the tools accumulated.

\subsection{Dirichlet Kernel Interpolation}

\begin{definition}
    The \emph{Dirichlet kernel with cutoff $f$} is the periodic function $K_f: \TT \to \RR$ defined by
    \[ K_f(t) = \frac{1}{2f + 1} \sum_{k = -f}^f e^{2\pi i k t}. \]
\end{definition}
The Dirichlet kernels are useful in our situation because they simultaneously satisfy two desirable properties: (1) they form a family of approximations to the identity as $f_c \to \infty$, and (2) they are low-pass for each fixed $f_c$.
Thus, these will be the basic building blocks of our dual certificates.

Concretely, a simple way to interpolate the sign pattern $v \in \mu^T$ is to declare that
\[ q(t) = \sum_{t_j \in T} \alpha_j K_{f_c}(t - t_j), \]
then solve the linear algebra problem arising for the $\alpha_j$ when we require that $q(t_j) = v_j$.
However, in this case we will run into problems ensuring that $|q(t)| < 1$ when $t \notin T$, since the tail of $K_{f_c}$ decays slowly, thus roughly speaking $q(t)$ has non-negligible contributions from many of the $t_j \in T$.
  
\subsection{Multi-Dirichlet Kernel Interpolation}

\begin{definition}
    The \emph{multi-Dirichlet Kernel with cutoffs $f\bm \gamma$} for $\bm\gamma \in \RR^p_{\geq 0}$ satisfying $\sum_{j = 1}^p \gamma_j = 1$ is
    \[ K_{f\bm\gamma}(t) = \prod_{j = 1}^p K_{\gamma_j f}(t). \]
    The constraint on $\bm \gamma$ ensures that $K_{f\bm\gamma}$ is indeed $f$-low-pass.
\end{definition}
The more spread out the coordinates of $\bm\gamma$ are, the less sharp $K_{f\bm\gamma}(t)$ is around $t = 0$, but the faster its tail decays.
Thus, introducing this extra parameter lets us control the trade-off between the quality of the interpolation vs.\ the control over the off-support magnitude.

\subsection{Interpolation with Derivative}

The remaining problem is rather technical: though our interpolation will now be exact and the magnitude $|q(t)|$ will be controlled away from each $t_j$, very near each $t_j$ we require $|q(t)|$ to achieve a local maximum, and there is no reason to think that the present construction will achieve this.
To this end, we simultaneously interpolate with the kernel and its derivative, and add an additional constraint that $q(t)$ has a critical point at each $t_j$ (the other aspects of our construction will then let us ensure that the critical point is in fact a local maximum of the magnitude).
We take
\[ q(t) = \sum_{t_j \in T} \left\{ \alpha_j K_{f_c \bm \gamma}(t - t_j) + \beta_j K_{f_c \bm \gamma}^\prime(t - t_j)\right\}, \]
and solve the larger linear algebra problem
\begin{align*}
  v_k &= \sum_{t_j \in T} \left\{ \alpha_j K_{f_c \bm \gamma}(t_k - t_j) + \beta_j K_{f_c \bm \gamma}^\prime(t_k - t_j)\right\} \\
  0 &= \sum_{t_j \in T} \left\{ \alpha_j K_{f_c \bm \gamma}^{\prime}(t_k - t_j) + \beta_j K_{f_c \bm \gamma}^{\prime\prime}(t_k - t_j)\right\}
\end{align*}
to obtain the coefficients $\alpha_j, \beta_j$.

This is the basic construction used in the proof of Theorem~\ref{thm:single-obs-recovery}.
There are two things left to verify:
\begin{enumerate}
\item The linear algebra problem above has a solution $(\bm \alpha, \bm\beta)$, which is fairly well-behaved.
\item $|q(t)| < 1$ for all $t \notin T$.
\end{enumerate}
The next two sections will outline the proofs of these statements.
Of particular interest is which proofs use which assumptions on $\Delta(T)$, and the relationships between the bounds involved.

\subsection{Solving for Interpolation Coefficients}
Defining matrices
\begin{align*}
  \bD^{(0)}_{jk} &= K_{f_c \bm \gamma}(t_k - t_j), \\
  \bD^{(1)}_{jk} &= K_{f_c \bm \gamma}^\prime(t_k - t_j), \\
  \bD^{(2)}_{jk} &= K_{f_c \bm \gamma}^{\prime\prime}(t_k - t_j),
\end{align*}
and the block matrix
\[ \bD = \left[\begin{array}{cc} \bD^{(0)} & \bD^{(1)} \\ \bD^{(1)} & \bD^{(2)} \end{array}\right], \]
the linear algebra problem for the interpolation coefficients may be written
\[ \bD\left[\begin{array}{c} \bm \alpha \\ \bm \beta \end{array}\right] = \left[\begin{array}{c} \bm v \\ \bm 0 \end{array}\right]. \]
There are then two things to verify.
\begin{lemma}
    $\bD$ is invertible.
\end{lemma}
\begin{lemma}
    Defining
    \[ \left[\begin{array}{c} \bm \alpha \\ \bm \beta \end{array}\right] \coloneqq \bD^{-1}\left[\begin{array}{c} \bm v \\ \bm 0 \end{array}\right], \]
    these $\bm\alpha$, $\bm\beta$ satisfy some $\ell^\infty$ bounds.
\end{lemma}
In the random sign pattern setting, we will have $\bv$ be a random variable, and therefore $\bm \alpha$ and $\bm\beta$ will be random as well.
In particular, if $\bv$ is close to Gaussian, then so are $\bm\alpha$ and $\bm\beta$, with covariance that we can write down in terms of $\bD^{-1}$.
There are many tools for bounding the supremum of a Gaussian vector, so we might hope to improve the bounds on $\bm\alpha$ and $\bm\beta$ in the random case.

This section of the argument in \cite{fernandez2016super} depends \emph{only} on the lower bound $f_c \geq 10^3$.
Thus, the most we can hope for from the above sketch is to reduce this number, or improve the resulting bounds on $\|\bm\alpha\|_\infty, \|\bm\beta\|_\infty$.
The latter is more helpful, since it feeds into the off-support bounds where it plays an important role.

\subsection{Proving Off-Support Bound}

\section{Random Sign Patterns}


\bibliographystyle{unsrt}
\bibliography{main}

\end{document}
%%% Local Variables:
%%% mode: latex
%%% TeX-master: t
%%% End:
